\documentclass{article}
\usepackage{latexTools}
\title{A Theorem on Bilinear Forms}
\author{D Walters}
\begin{document}
\maketitle
    \begin{theorem}
 Assume \(F\) is a bilinear form. Let \(A\) be the diagonal matrix representing \(F\), and assume it has no \(0\)'s on it's diagonal.
 Then \(F\) is positive definite if and only if the diagonal contains only strictly positive elements.
    \end{theorem}
    \begin{proof}
      Let the diagonal elements of the matrix be \(a_{1}, \ldots,a_{n}\) such that the matrix has the form
      \begin{equation}
\begin{pmatrix}
  a_{1} & 0 & 0 & 0\\
  0 & a_{2} & 0 & 0 \\
  \vdots & \vdots & \ddots & \vdots \\
  0 & 0 & 0 & a_{n}
\end{pmatrix}
      \end{equation}

      First assume that \(a_{1},\ldots a_{n} > 0\), and let \(x\) be an arbitrary non-zero vector such that
      \begin{align}
        x&=\sum_{i=1}^{n}\lambda_{i}e_{i}\\
      \intertext{where \(\lambda_{1},\ldots, \lambda_{n}\in \IR\) and are not all \(0\). Then we have that}
        F(x,x) &= \sum_{i=1}^{n}\lambda_{i}^{2}a_{i} \geq 0.\\
                 \intertext{However, \(F(x,x)=0\) if and only if all \(\lambda_{i}=0\), so we can conclude that}
        F(x,x) &> 0
      \end{align}
      and therefore that \(F\) is positive definite.

      Now assume that some \(a_{k} < 0\), then choose \(x=e_{k}\). Then,
      \begin{align}
F(x,x)=F(e_{k},e_{k})=a_{k}<0
      \end{align}
      so \(F\) is not positive definite.

    \end{proof}
\end{document}
